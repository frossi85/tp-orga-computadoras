\documentclass{article}

%% PAQUETES

% Paquetes generales
\usepackage[margin=2cm, paperwidth=210mm, paperheight=297mm]{geometry}
\usepackage[spanish]{babel}
\usepackage[utf8]{inputenc}
\usepackage{gensymb}

% Paquetes para estilos
\usepackage{textcomp}
\usepackage{setspace}
\usepackage{colortbl}
\usepackage{color}
\usepackage{color}
\usepackage{upquote}
\usepackage{xcolor}
\usepackage{listings}
\usepackage{caption}
\usepackage[T1]{fontenc}
\usepackage[scaled]{beramono}

% Paquetes extras
\usepackage{amssymb}
\usepackage{float}
\usepackage{graphicx}
\usepackage{url}
\usepackage[toc,page]{appendix}
\usepackage{mips}


%% Fin PAQUETES


% Definición de preferencias para la impresión de código fuente.
%% Colores
\definecolor{gray99}{gray}{.99}
\definecolor{gray95}{gray}{.95}
\definecolor{gray75}{gray}{.75}
\definecolor{gray50}{gray}{.50}
\definecolor{keywords_blue}{rgb}{0.13,0.13,1}
\definecolor{comments_green}{rgb}{0,0.5,0}
\definecolor{strings_red}{rgb}{0.9,0,0}

%% Caja de código
\DeclareCaptionFont{white}{\color{white}}
\DeclareCaptionFont{style_labelfont}{\color{black}\textbf}
\DeclareCaptionFont{style_textfont}{\it\color{black}}
\DeclareCaptionFormat{listing}{\colorbox{gray95}{\parbox{16.78cm}{#1#2#3}}}
\captionsetup[lstlisting]{format=listing,labelfont=style_labelfont,textfont=style_textfont}

\lstset{
	aboveskip = {1.5\baselineskip},
	backgroundcolor = \color{gray99},
	basicstyle = \ttfamily\footnotesize,
	breakatwhitespace = true,   
	breaklines = true,
	captionpos = t,
	columns = fixed,
	commentstyle = \color{comments_green},
	escapeinside = {\%*}{*)}, 
	extendedchars = true,
	frame = lines,
	keywordstyle = \color{keywords_blue}\bfseries,
	language = C,                       
	numbers = left,
	numbersep = 5pt,
	numberstyle = \tiny\ttfamily\color{gray50},
	prebreak = \raisebox{0ex}[0ex][0ex]{\ensuremath{\hookleftarrow}},
	rulecolor = \color{gray75},
	showspaces = false,
	showstringspaces = false, 
	showtabs = false,
	stepnumber = 1,
	stringstyle = \color{strings_red},                                    
	tabsize = 2,
	title = \null, % Default value: title=\lstname
	upquote = true,                  
}

%% FIGURAS
\captionsetup[figure]{labelfont=bf,textfont=it}
%% TABLAS
\captionsetup[table]{labelfont=bf,textfont=it}

% COMANDOS

%% Titulo de las cajas de código
\renewcommand{\lstlistingname}{Código}
%% Titulo de las figuras
\renewcommand{\figurename}{Figura}
%% Titulo de las tablas
\renewcommand{\tablename}{Tabla}
%% Referencia a los códigos
\newcommand{\refcode}[1]{\textit{Código \ref{#1}}}
%% Referencia a las imagenes
\newcommand{\refimage}[1]{\textit{Imagen \ref{#1}}}

%% APENDICES
\addto\captionsspanish{
	\renewcommand\seename{Apéndices}
	\renewcommand\appendixname{Apéndices}
	\renewcommand\appendixpagename{Apéndices}
}


\begin{document}


% TÍTULO, AUTORES Y FECHA
\begin{titlepage}
	\vspace*{\fill}
	\begin{center}
		\Huge \textit{``Conjunto de instrucciones MIPS''} \\
		
		\bigskip\bigskip\bigskip\bigskip\bigskip

		\Large Federico Colangelo, Padrón Nro. 89.869 \\
		\large \textit{federico.colangelo@semperti.com} \\ \medskip
		\Large Facundo Rossi, Padrón Nro. 86.707 \\
		\large \textit{frossi85@gmail.com} \\ \medskip
		\Large Federico Martín Rossi, Padrón Nro. 92.086 \\
		\large \textit{federicomrossi@gmail.com} \\

		\bigskip\bigskip\bigskip\bigskip\bigskip\bigskip\bigskip

		\large 1er. Cuatrimestre 2013 \\ \smallskip
		\large 66.20 Organización de Computadoras \\ \smallskip
		\large Facultad de Ingeniería, Universidad de Buenos Aires \\ \smallskip

		\date{}
	\end{center}
	\vspace*{\fill}
\end{titlepage}
\newpage



% ÍNDICE
\tableofcontents
\newpage




% Introducción
\section{Introducción}
	
	[ Colocar texto aquí ]
\bigskip



% Implementación
\section{Implementaciones}
	
	[ Colocar texto aquí ]




% Algoritmo Bubblesort (en lenguaje C)
\subsection{Algoritmo \textit{Bubblesort} (en lenguaje C)}

	[ Colocar texto aquí]

% Código
\lstset{ language = C } % Cambiamos el lenguaje para que parsee en C
\lstinputlisting[label=codeBSh,caption=``bubblesort.h'']{../Codigo/c/bubblesort.h} 
\bigskip\bigskip


% Código
\lstset{ language = C } % Cambiamos el lenguaje para que parsee en C
\lstinputlisting[label=codeBSc,caption=``bubblesort.c'']{../Codigo/c/bubblesort.c} 
\bigskip\bigskip




% Algoritmo Shellsort (en lenguaje C)
\subsection{Algoritmo \textit{Shellsort} (en lenguaje C)}

	[ Colocar texto aquí]

% Código
\lstset{ language = C } % Cambiamos el lenguaje para que parsee en C
\lstinputlisting[label=codeSSh,caption=``shellsort.h'']{../Codigo/c/shellsort.h} 
\bigskip\bigskip


% Código
\lstset{ language = C } % Cambiamos el lenguaje para que parsee en C
\lstinputlisting[label=codeSSc,caption=``shellsort.c'']{../Codigo/c/shellsort.c} 
\bigskip\bigskip




% Conclusiones
\section{Conclusiones}


\bigskip





% Citas bibliográficas.
\begin{thebibliography}{99}

	\bibitem{HEN00} J. L. Hennessy and D. A. Patterson, ``Computer Architecture. A Quantitative
	Approach,'' 4th Edition, Morgan Kaufmann Publishers, 2000.

	\bibitem{GXEMUL} The NetBSD project, \url{http://www.netbsd.org/}

	\bibitem{BS} Bubblesort, \url{http://en.wikipedia.org/wiki/Bubble_sort}

	\bibitem{SS} Shellsort, \url{http://en.wikipedia.org/wiki/Shell_sort}

	\bibitem{TIME} time man page, \url{http://unixhelp.ed.ac.uk/CGI/man-cgi?time}

	\end{thebibliography}



\newpage


% Apendices
\begin{appendices}

\bigskip\bigskip

% Implementación completa en lenguaje C
\section{Implementación completa en lenguaje C}


\subsection{\textit{tp0.c}. Implementación del main del programa}
% Código
\lstset{ language = C } % Cambiamos el lenguaje para que parsee en C
\lstinputlisting[label=codeTP0cfull,caption=``tp0.c'']{../Codigo/c/tp0.c} 
\bigskip\bigskip

\subsection{\textit{bubblesort.h}. Declaración de la librería Bubblesort}
% Código
\lstset{ language = C } % Cambiamos el lenguaje para que parsee en C
\lstinputlisting[label=codeBShfull,caption=``bubblesort.h'']{../Codigo/c/bubblesort.h} 
\bigskip\bigskip

\subsection{\textit{bubblesort.c}. Definición de la librería Bubblesort}
% Código
\lstset{ language = C } % Cambiamos el lenguaje para que parsee en C
\lstinputlisting[label=codeBScfull,caption=``bubblesort.c'']{../Codigo/c/bubblesort.c} 
\bigskip\bigskip

\subsection{\textit{shellsort.h}. Declaración de la librería Shellsort}
% Código
\lstset{ language = C } % Cambiamos el lenguaje para que parsee en C
\lstinputlisting[label=codeSShfull,caption=``shellsort.h'']{../Codigo/c/shellsort.h} 
\bigskip\bigskip

\subsection{\textit{shellsort.c}. Definición de la librería Shellsort}
% Código
\lstset{ language = C } % Cambiamos el lenguaje para que parsee en C
\lstinputlisting[label=codeSScfull,caption=``shellsort.c'']{../Codigo/c/shellsort.c} 
\bigskip\bigskip

\subsection{\textit{fileloader.h}. Declaración de la librería File Loader}
% Código
\lstset{ language = C } % Cambiamos el lenguaje para que parsee en C
\lstinputlisting[label=codeFLhfull,caption=``fileloader.h'']{../Codigo/c/fileloader.h} 
\bigskip\bigskip

\subsection{\textit{fileloader.c}. Definición de la librería File Loader}
% Código
\lstset{ language = C } % Cambiamos el lenguaje para que parsee en C
\lstinputlisting[label=codeFLcfull,caption=``fileloader.c'']{../Codigo/c/fileloader.c} 
\bigskip\bigskip




% Implementación completa en assembler MIPS
\section{Implementación completa de Shellsort en assembly MIPS}


\subsection{\textit{main.c}. Implementación del main del programa}
% Código
\lstset{ language = C } % Cambiamos el lenguaje para que parsee en C
\lstinputlisting[label=codeTP0afull,caption=``main.c'']{../Codigo/assembly/main.c} 
\bigskip\bigskip

\subsection{\textit{shellsort.S}. Implementación del algoritmo Shellsort}
% Código
% \lstset{ language = [mips]Assembler} % Cambiamos el lenguaje para que parsee en MIPS
% \lstinputlisting[label=codeSSafull,caption=``shellsort.S'']{../Codigo/assembly/shellsort.S} 
% \bigskip\bigskip

\subsection{\textit{compare.S}. Implementación del comparador de palabras}
% Código
% \lstset{ language = [mips]Assembler} % Cambiamos el lenguaje para que parsee en MIPS
% \lstinputlisting[label=codeCMPafull,caption=``compare.S'']{../Codigo/assembly/compare.S} 
% \bigskip\bigskip

\subsection{\textit{swap.S}. Implementación del comparador de palabras}
% Código
% \lstset{ language = [mips]Assembler} % Cambiamos el lenguaje para que parsee en MIPS
% \lstinputlisting[label=codeSWAPafull,caption=``swap.S'']{../Codigo/assembly/swap.S} 
% \bigskip\bigskip



\end{appendices}

\end{document}
